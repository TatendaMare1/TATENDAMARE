\documentclass[a4paper,11pt]{report}
\usepackage{geometry}
\usepackage{graphicx}
\usepackage{amsmath}
\usepackage{amsfonts}
\usepackage{hyperref}
\usepackage{fancyhdr}
\usepackage{setspace}

% Page margins
\geometry{
	left=1.5cm,
	right=1.5cm,
	top=1.5cm,
	bottom=1.5cm
}

% Roman and Arabic page numbering
\pagenumbering{roman}

\title{BACHELOR OF SCIENCE HONOURS DEGREE IN DATA SCIENCE\\
	DATA DRIVEN NUTRITIONAL ANALYSIS OF LOCAL FOODS:PREDICTING DIETARY PATTERNS AND MALNUTRITION RISKS IN SOUTHERN AFRICA}
\author{TATENDA MARE\\
	REGISTRATION NUMBER: R219488A\\
	DEGREE PROGRAMME: HONOURS IN DATA SCIENCE}
\date{SEPTEMBER 2024}

\pagestyle{fancy}
\fancyhf{}
\fancyfoot[C]{\thepage}

\begin{document}
	
	\maketitle
	
	
	\newpage
	
	\tableofcontents
	
	\newpage
	
	% Start Arabic numbering for chapters
	\pagenumbering{arabic}
	
	\chapter{INTRODUCTION}
	\section{Introduction}
Malnutrition has been a very prevalent issue of critical concern. Despite the goals that the World has set in terms of solving the issue of malnutrition it is in no way close to achieving the goals that it had set in this regard. This issue is very critical in southern Africa. Of all the countries that are suffering from malnutrition most of the countries are located in Africa. Malnutrition is a condition where the body is not receiving enough nutrients for the body or too much food is consumed that also inhibits the body from functioning normally. A study into the issue of malnutrition is very important as it aims to help the vulnerable people within the communities in these regions. These people are the children under the age of 5 and pregnant women. The aim of this study is to analyse the dietary patterns in Southern Africa and how they are affecting the health of the community. It aims to draw the relationships between dietary patterns and malnutrition risks and what nutrient deficiencies are a product of these risks. After the identification of the risks the goal is for policy makers and other relevant stakeholders to make informed decisions and policies based on the more detailed information that this study will offer.
	\par This chapter looks into the background of the study, looks into the objectives of the project, aims of the study, research questions, delimitations and the limitations of the study.
	
	\section{Background of Study}
Malnutrition, despite popular beliefs, has many forms. The forms of malnutrition include undernutrition, macro and micro-nutrient deficiencies and over nutrition. Under nutrition is the condition when the body is not receiving enough food that the body requires in order to function well. These include wasting, stunting and underweight. As stated by (Global Nutrition) in Southern Africa the prevalence of stunting is 23.3\% which is higher than that of the global average of 22\% . Over nutrition  causes disease such as obesity and cardiovascular diseases.As shown by (Global Nutrition) “an average of 13.9\% of adults of women live with diabetes, compared to 11.1\% of men , meanwhile 41.5 of women live with diabetes as compared to 17.1\% of men who live with obesity. As stated by (Global Nutrition,n.d) the prevalence of wasting is 3.2\% as compared to the global average of 6.7\%.Wasting is an example of a macronutrient deficiency as protein is needed for cell growth. According to (Galani,Orfila,Gong,2020) “Estimates of their prevalence are among the highest in East and South Africa: iron-deficient anaemia affected 26 to 31\% of women of reproductive age, and deficiencies up to 53\%, 36\%, 66\%, 75\% and 62\% for vitamin A, iodine, zinc, calcium and selenium, respectively, were measured in populations of these regions.  ”Anaemia has been shown to be one of the most problematic deficiencies as it affects pregnant women and their unborn children in the interim.\par
	The progress in developed countries has been modest. In North America they are on the way of curbing malnutrition in most of their countries, that is overweight, stunting and wasting. However none of the countries are anywhere close to solving the problem of anaemia in women of reproductive age, low birth weight, diabetes among men and women and obesity in men and women (Global Nutrition,nd).In the United States of America, for example, they have food programmes that assist the people by offering nutrition programmes such as SNAP(FRAC,2023).The community also assists by offering initiatives such as food banks where those who cannot afford food can you and get a warm nutritional meal. However some of these programmes do not reach people equally, households that have children are more likely to face hunger as well as minority communities (Pathak,et al, 2022).In Asia their goal towards their goal to reduce global nutrition has been modest as well. Despite their large populations issues like stunting and wasting are on track. However low birth weight and obesity is not on track (Global Nutrition, nd). In China they implemented a National Nutritional Plan in response to the Sustainable Goal Developments. This included a rapid increase in food accessibility, diversity and affordability. The main contributor to this increase is China`s growing economic status. However problems such as triple burden of malnutrition are still affecting the country. (Cai, et al,2022).\par
	Food in the southern region of Africa is diverse. Most of the crops that are grown in Southern Africa are cereals . The main cereal being maize. A number of staple dishes are based on maize such as sadza from Zimbabwe, oshitsima from Namibia and Nshima from Zambia just to mention a few. This shows that most meals are centred around carbohydrates. Proteins are usually consistent of Beef and poultry and they have also include vegetables that are grown locally such as soya beans and leafy greens. In other countries that are close to coastal areas such as Malawi they have fish as part of their main diet. The diet may also change seasonally depending on what is available during that season. Culture plays an influential part on what food a community consumes. In the rural areas their diet is linked to traditional diets such as maize and millet. Their diets are a reflection of ecological and cultural changes. However in the urban area their diet is diverse but is limited by purchasing power. Rural communities prioritize food security through locally sourced foods whereas in the urban regions they rely on processed foods that are rich in sugars and unsaturated fats (Perico, 2024).\par
	There is a relationship between poverty and malnutrition. Poverty creates an instability that fuels malnutrition. Poverty and undernutrition and macro-micro nutrient deficiencies are heavily linked. The lack of food leads to lack of carbohydrates, proteins and other nutrients which in turn leads to deficiencies (Siddiqui,et al, 2020).” There is an increasing global trend towards urbanization. In general, there are less food access issues in urban than rural areas, but this “urban advantage” does not benefit the poorest who face disproportionate barriers to accessing healthy food and have an increased risk of malnutrition ”(Vilar-Compte , et al, 2021)\par
African diets are diverse and consist of foods that are rich in fibre which are grains, vegetables and many others. However with the introduction of urbanisation foods that are processed have been introduced to African countries and his has lead to the appearance of non-communicable diseases that include cardiovascular diseases and obesity. These foods are less nutritional than the foods that are sourced locally. Understanding how changes in the food system affect health and diet is crucial for Africa (Okoye, et al,2024). “Although agriculture provides the food that humanity requires, agricultural landscapes are becoming increasingly simplified because the variety of crops that are grown on farms is declining and is threatening agricultural biodiversity, while at the same time there is a trend towards the homogenisation of diets”( Schönfeldt,2020).Many of the dietary assessments are based of the western community meaning that they are not adapted to African communities and how they consume their food(Abdelradi, et al,2021).Due to urbanization, diets in Africa are shifting towards processed and unhealthy food which may not be captured by traditional assessment techniques(Vos, et al,2023). Due to logistical challenges, a lack of infrastructure, and restricted access to technology, collecting accurate data in rural and isolated locations can be challenging. This may result in data gaps and an insufficient comprehension of eating habits (Vos, et al,2021)\par
	Reducing malnutrition has been a key priority in Africa. The goal is that by 2030 all hunger and malnutrition is reduced. However the number of countries in Africa that have managed to hit that target are fewer than imagined and worse in Southern Africa. Going forward there is a need for policies that are leverage science and make use of digital technology (World Bank,2020).Research and extension, safety nets, consumption subsidies are policies that have been implemented and these policies have given rise to purchase power which enables consumers to buy foods that have the nutrients they are lacking (Azomahou, et al, 2022).Malnutrition is a double burden in the sense that there are place where undernutrition and obesity coexist and it is very hard to implement policies that can help these areas (WHO, 2019).Economic growth that is continuous is vital for poverty reduction and in turn in growth of the general nutrition of people in a country. Improving nutrition should be of high national importance. Many countries in Sub Saharan Africa are facing challenges in addressing these issues. One key thing that has been noted is that nutritionists have not been placed in areas where they can assist in making of these policies (Babu, 2001).In some countries adolescents and adults may not be a key focus on these policies even with efforts to prioritize them(Mukanu, et al,2023).\par
	Reducing malnutrition has been a key priority in Africa. The goal is that by 2030 all hunger and malnutrition is reduced. However the number of countries in Africa that have managed to hit that target are fewer than imagined and worse in Southern Africa. Going forward there is a need for policies that leverage science and make use of digital technology (World Bank,2020).Research and extension, safety nets, consumption subsidies are policies that have been implemented and these policies have given rise to purchase power which enables consumers to buy foods that have the nutrients they are lacking (Azomahou, et al, 2022).Malnutrition is a double burden in the sense that there are places where undernutrition and obesity coexist and it is very hard to implement policies that can help these areas (WHO, 2019).Economic growth that is continuous is vital for poverty reduction and in turn in growth of the general nutrition of people in a country. Improving nutrition should be of high national importance. Many countries in Sub Saharan Africa are facing challenges in addressing these issues. \par
Data driven strategies then come into play in assisting with the problems that are faced when it comes to implementing policies that are suitable for alleviating nutrient and food deficiencies. Analysis of datasets to identify trends and patterns is used to come up with malnutrition risks and in turn policy makers can make policies that are tailor made for that specific area.. Predictive modelling is going to be applied that is going to predict future dietary patterns and associated malnutrition risks based on food production  data from FAO.There is an urgent need for ending all forms of malnutrition by 2030 as pledged in SDGs (WHO,2019).This study aims to fill the gaps by studying dietary patterns that may have been overlooked in broader analyses. It aims to identify nutritional content of locally available food and pin pointing specific micro nutrient defeciencies. With the use of machine learning and sophisticated analytics, the study is able to predict future food patterns and risks of malnutrition. With the use of this predictive capabilities, stakeholders may foresee possible problems and take action before they become more serious.
	
	\section{Problem Statement}
	Malnutrition is a significant public health concern in Southern Africa that is largely driven by compromised dietary intake influenced by local food systems.This study is evaluating the high rates of malnutrition in southern Africa among the more vulnerable demographics. There are many aspects of local food systems that this study focuses on. These aspects include diversity of crops, urbanisation and demographic changes, economic inequalities and nutrition and health. The study will focus on children, women and the elderly.. The dietary patterns that this study will be focusing on are the variations in nutrition value of the food, nutritional diversity, the effect of seasonal patterns on the diet, cultural influences on dietary patterns and how the economy affects the diet.Malnutrition causes major health implications that can affect the way of living in communities.In children it affects their cognitive development and their well being (UNICEF, 2021). “Also, malnutrition increases health care costs, reduces productivity, and slows economic growth, which can perpetuate a cycle of poverty and ill-health. (WHO,2024)”. Statistics show that the situation is very dire as compared to other regions in the globe. This shows that a study in this area is important and that it is a topic that must be addressed. It is shown that in the world about 5.6 million kids are dying from starvation before they reach the age of 5. 80\% of these deaths are occurring in Sub-Saharan Africa and Asia (Shipanga, et al,2023).If the problem of malnutrition is not solved a lot of critical issues will arise. Among these issues we have the rise in healthcare costs, a decline in the economy and in the process an increase in poverty. By analysing and understanding  the dietary patterns and the nutritional value of local food, this research will provide critical insights for healthcare providers, policy makers and other relevant stakeholders.
	
	\section{Aims}
	TThe main aim of the study is to analyse the dietary patterns of a community and the nutritional value of the food to understand the relationship between the diet and the prevalent malnutrition risks.
	
	\section{Objectives}
	\begin{enumerate}
		\item Evaluate regional effects of food security: Evaluate the effect of malnutrition in one part of southern Africa on the rest of the countries in the region.
		\item Plotting food demand and supply in southern Africa: Determine the deficiency related to the food that is not grown in the region
		\item Forecasting food security trends in southern Africa: Analyze and predict food availability.
		\end {enumerate}
	
	\section{Research Questions}
	\begin{enumerate}
		\item What are the dietary patterns that are common among vulnerable groups in Southern Africa and how do they affect the health of these people?
		\item How is the nutritional status of the food being consumed by people in Southern Africa relating to the diseases they are suffering from?
		\item How is the economic status of a region related to the health status of people in that region?
		\item What is the effect of economic status on the availability of food and what food is being grown locally? 
		\item What are the most common macro and micro nutrient deficiencies in children, women and the elderly?
		\item What is the effect of culture on the dietary patterns of a community?
		
	\end{enumerate}
	
	\section{Significance of Study}
	The creation of a model that can predict dietary patterns and malnutrition risks can help policy makers create policies that are well suited for the location and includes factors that affect the dietary patterns of Southern Africa such as the culture ,seasonal variations and socioeconomic influences through the predictive modelling. Out of the 29 low level income countries in the world 23 of these are located in Sub Saharan Africa. A lot of these countries have the double burden of malnutrition and that is the overlap between over nutrition and undernutrition in the same community or even triple burden which includes stunted growth in children under five, anaemia in women of reproductive age and obesity in adult women. Of the 41 countries that were recorded to have this problem in the world in 2018, 30 of them were found in Africa. Most studies look at undernutrition and often overlook the issue of overnutrtion when it comes to studying malnutrition in Africa (Delisle, et al, 2021). There is a high morbidity and mortality rate in Sub Saharan Africa as compared to anywhere else on the globe.  “Malnutrition has been a public health challenge in Sub-Saharan Africa that has not received enough attention”. There is a need to prioritize the prevention and control of malnutrition in Africa, together with practical recommendations (Owolade, et al,2020). This is why this study is very important. The people who would benefit from these studies are policymakers, health professionals, educational institutions and the whole of Southern Africa as a whole. They gain insights on other factors that could have been missed the causes of malnutrition and they gain an understanding of the trends  apply policies that could assist the southern region of Africa.
	
	\section{Delimination(Scope)}
	This study will mainly focus on vulnerable populations. The vulnerable populations that are of main focus are children under the age of 5, pregnant women  and the elderly in Southern Africa and will not be focusing on older adults and non-pregnant adults. Countries that will be of main focus are those found within the southern region. These countries are Zimbabwe, Zambia, South Africa, Namibia, Mozambique, Lesotho, Eswatini, Botswana and Angola. Any countries that are outside of the southern region will be excluded from the study. Children who suffer from chronic illnesses such as cancer will not be included in this study and statistics that are related to chronic illnesses will not be added to the study. The variables that this study will be making use of are food availability, dietary patterns, socio-economic status, food security indicators, nutritional status and malnutrition deficiencies. However climate change will not be a variable that will be focused on.
	
	\section{Limitations}
Sample size may be limited to a specific number of participants and may not take a broader perspective from the whole community. Statistical methods may be applied to counter the issue of variability. There will be a need to get diverse representations of the community. Data on dietary intake may be affected by people giving too little information or giving too much information that is irrelevant. This issue can be solved by the use of predictive modelling.The traditional way of analysis may not be able to capture complex relationships between variables but this is something that can be solved through the use of predictive models that can analyse large datasets and understand the complex relationships between the variables. Findings from one community may not be applicable to another community but by making use of machine learning models they can be trained on diverse datasets meaning a model can be created that can be trained on data from different regions.
\section{Definition of terms}
\textbf{Malnutrition}\newline
	Malnutrition is a condition whereby the the body is note receiving enough nutrients or is receiving too many nutrients.\newline
	\textbf{Overnutrition}\newline
	Overnutrition is a condition caused by consuming too much food.\newline
	\textbf{Under Nutrition}\newline
	Under nutrition is a condition where the body is not receiving enough nutrients that it needs to function.\newline
	\textbf{Macronutrien Deficiencies}\newline
	Macronutrient deficiencies are defeciencies where the body is not receiving enough macro nutrients such as proteins and carbohydrates\newline
	\textbf{Micronutrient Defeciencies}\newline
	Micronutrient deficiencies  are caused by a lack of vitamins and minerals in the body.\newline
	\textbf{Dietary Patterns}\newline
	Dietary Patterns are trends that are observed by studying the consumption of food and beverages overtime.\newline
	\texbf{Food Security}\newline
	Food Security is the state in which people have access to all the food they need for leading a healthy and active life.
	\section{Concluding Remarks}
In conclusion this study aims to analyse the dietary patterns and the nutritional status of the food that the community is consuming and understanding what diseases they are prone to based off of their diets. A relationship between the nutritional value of food and the type of deficiencies has been noted and that there are factors such as socio-economic factors, cultural influences, food production levels and food security indicators. By providing valuable insights this study aims to assist in providing policies that are tailored made for Southern Africa and in turn help curb the issue of malnutrition and improve the health of the people.
	,
	\bibliographystyle{plain}
	\bibliography{references} % Use your .bib file for references
	\begin{thebibliography}
	\bibitem{Abdelradi2021}
	Abdelradi, F., Admassie, A., Adjaye, J.A., Ayieko, M., Badiane, O., Glatzel, K., Hendriks, S., Mbaye, M.S., Mengoub, F.E., Ramadan, R., Olofinbiyi, T., Sibanda, S. (2021). Policy options for food systems transformation in Africa - from the perspective of African universities and think tanks. \textit{FSSBrief Policy Options Food Systems Transformation Africa}. \url{https://fss2021.org} [Accessed 24 September 2024].
	
	\bibitem{Cai2022}
	Cai, L., Hu, X., Liu, S., Wang, L., Wang, X., Tu, H., Tong, Y. (2022). China is implementing the national nutrition plan of action. \textit{Front Nutr}, 9. \url{https://doi.org/10.3389/fnut.2022.983484} [Accessed 26 September 2024].
	
	\bibitem{ClevelandClinic}
	Cleveland Clinic. (n.d.). Malnutrition: Definition, Causes, Symptoms Treatment. \url{https://clevelandclinic.org} [Accessed 22 September 2024].
	
	\bibitem{Delisle2021}
	Delisle, H., Faber, M., Revault, P. (2021). Evidence-based strategies needed to combat malnutrition in Sub-Saharan countries facing different stages of nutrition transition. \textit{Public Health Nutr}, 24(12), 3577–3580. \url{https://doi.org/10.1017/S1368980021001221} [Accessed 24 September 2024].
	
	\bibitem{FoodResearch}
	Food and Research Action Center. (2024). Hunger Poverty in America. \url{https://frac.org} [Accessed 26 September 2024].
	
	\bibitem{Galani2020}
	Galani, Y.J.H., Orfila, C., Gong, Y.Y. (2020). A review of micronutrient deficiencies and analysis of maize contribution to nutrient requirements of women and children in Eastern and Southern Africa. \textit{Critical Reviews in Food Science and Nutrition}, 62(6), 1568–1591. \url{https://doi.org/10.1080/10408398.2020.1844636} [Accessed 23 September 2024].
	
	\bibitem{GlobalNutrition}
	Global Nutrition Report. (n.d.). The burden of malnutrition at a glance. \url{https://globalnutritionreport.org/resources/nutrition-profiles/africa/southern-africa/} [Accessed 24 September 2024].
	
	\bibitem{Kanter2023}
	Kanter, R., Kennedy, G., Boza, S. (2023). Local, traditional and indigenous food systems in the 21st century to combat obesity, undernutrition and climate change. \textit{Front. Sustain. Food Syst.}, 7:1195741. \url{https://doi.org/10.3389/fsufs.2023.1195741} [Accessed 24 September 2024].
	
	\bibitem{Monterosa2020}
	Monterosa, E.C., Frongilo, E.A., Vandevijvere, S., Drewnowski, A., Pee, S. (2020). Sociocultural Influences on Food Choices and Implications for Sustainable Healthy Diets. \textit{Food and Nutrition Bulletin}, 41(2). \url{https://doi.org/10.1177/0379572120975874} [Accessed 22 September 2024].
	
	\bibitem{Mulala2021}
	Mulala, J. (2021). A community approach to fight malnutrition. \textit{UNICEF West and Central Africa}. \url{https://unicef.org} [Accessed 24 September 2024].
	
	\bibitem{Okoye2024}
	Okoye, C.O., Enechi, C.O., Olanipekun, I.A., Obiefule, U.N., Asumadu-Boateng, G.K., Emejuru, S.C., Onwe, R.K., Ezehmalu, J.A., Ayanwunmi, B.T. (2024). Impact of Food Systems Transformation on Dietary Patterns and Public Health in Africa: A Mini Review. \textit{Asian Journal of Food Research and Nutrition}, 3(3), 747-756. \url{https://journalajfrn.com/index.php/AJFRN/article/view/171/346?im−XlGeKXxH=15752327343989950248} [Accessed 23 September 2024].
	
	\bibitem{Owolade2022}
	Owolade, A.J., Abdullateef, R.O., Adesola, R.O., Olaloye, E.D. (2022). Malnutrition: An underlying health condition faced in sub-Saharan Africa: Challenges and recommendations. \textit{Annals of Medicine and Surgery}, 82. \url{https://doi.org/10.1016/j.amsu.2022.104769} [Accessed 25 September 2024].
	
	\bibitem{Pathak2022}
	Pathak, A., Richards, R., Jarsulic, M. (2022). The United States Can End Hunger and Food Insecurity for Millions of People. \textit{Center for American Progress}. \url{https://americanprogress.org} [Accessed 26 September 2024].
	
	\bibitem{Perio2024}
	Perio, P. (2024). Social and Cultural Factors Influencing Food Choices in Urban vs. Rural Africa. \textit{African Journal of Food Science and Technology}. [Accessed 23 September 2024].
	
	\bibitem{Schonfeldt2020}
	Schonfeldt, H. (2020). Food cultures of southern Africa. \textit{South African Journal of Science}, 116(3/4). \url{https://doi.org/10.17159/sajs.2020/7555} [Accessed 22 September 2024].
	
	\bibitem{Shipanga2024}
	Shipanga, V., Kappas, M., Wyas, D. (2024). The status of research on malnutrition among children under 5 years in Southern Africa: a systematic review. \textit{International Journal of Community Medicine and Public Health}, 11(1), 363-370. \url{https://dx.doi.org/10.18203/2394-6040.ijcmph2023415} [Accessed 24 September 2024].
	
	\bibitem{Siddiqui2020}
	Siddiqui, F., Salam, R.A., Lassi, Z.S., Das, J.K. (2020). The Intertwined Relationship Between Malnutrition and Poverty. \textit{Front Public Health}. [Accessed 23 September 2024].
	
	\bibitem{Vos2024}
	Vos, K., Janssens, C., Jacobs, L., Campforts, B., Boere, E., Kozicka, M., Leclere, D., Havlik, P., Hemerijckx, L., Rompaey, A., Maertens, M., Govers, G. (2024). African food system and biodiversity mainly affected by urbanization via dietary shifts. \textit{Nat Sustain}, 7, 869–878. \url{https://doi.org/10.1038/s41893-024-01362-2} [Accessed 24 September 2024].
	
	\bibitem{WHO2019}
	World Health Organization. (2019). Strategic Plan to Reduce the Double Burden of Malnutrition in the African Region: 2019–2025. \url{https://who.int} [Accessed 24 September 2024].
	
	\bibitem{WHO2024}
	World Health Organisation. (2024). Malnutrition. Fact sheets - Malnutrition. [Accessed 22 September 2024].
	\end{thebibliography}
	
	
\end{document}